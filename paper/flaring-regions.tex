\documentclass[namedreferences]{solarphysics}

\usepackage{pythontex}
% % % % % % % % % % % % % % % % % % % % % % % % % % % % % % % % % %
% PythonTeX Bug Fix % % % % % % % % % % % % % % % % % % % % % % % %
% % % % % % % % % % % % % % % % % % % % % % % % % % % % % % % % % % 
% pytexbug fix for context in customcode.
\makeatletter
\renewenvironment{pythontexcustomcode}[2][begin]{%
    \VerbatimEnvironment
    \Depythontex{env:pythontexcustomcode:om:n}%
    \ifstrequal{#1}{begin}{}{%
        \ifstrequal{#1}{end}{}{\PackageError{\pytx@packagename}%
            {Invalid optional argument for pythontexcustomcode}{}
        }%
    }%
    \xdef\pytx@type{CC:#2:#1}%
    \edef\pytx@cmd{code}%
    % PATCH \def\pytx@context{}%
    \pytx@SetContext
    % END PATCH
    \def\pytx@group{none}%
    \pytx@BeginCodeEnv[none]}%
{\end{VerbatimOut}%
\setcounter{FancyVerbLine}{\value{pytx@FancyVerbLineTemp}}%
\stepcounter{\pytx@counter}%
}%
\makeatother
% % % % % % % % % % % % % % % % % % % % % % % % % % % % % % % % % %

\setpythontexcontext{figurewidth=\the\columnwidth, textwidth=\the\textwidth}
\newcommand{\includepgf}[1]{\IfFileExists{#1}{\input{#1}}{}}

\usepackage[hyperref,optionalrh,showbiblabels]{spr-sola-addons} % For Solar Physics 
%\usepackage[optionalrh]{spr-sola-addons} % For Solar Physics 
%\usepackage{epsfig}          % For eps figures, old commands
\usepackage{graphicx}        % For eps figures, newer & more powerfull
%\usepackage{courier}         % Change the \texttt command to courier style
%\usepackage{amssymb}        % useful mathematical symbols
\usepackage{color}           % For color text: \color command
\usepackage{breakurl}        % For breaking URLs easily trough lines
\def\UrlFont{\sf}            % define the fonts for the URLs

% General definitions
% please place your own definitions here and don't use \def but
% \newcommand{}{} or 
% \renewcommand{}{} if it is already defined in LaTeX

\newcommand{\BibTeX}{\textsc{Bib}\TeX}
\newcommand{\etal}{{\it et al.}}

% Definitions for equations
\renewcommand{\vec}[1]{{\mathbfit #1}}
\newcommand{\deriv}[2]{\frac{{\mathrm d} #1}{{\mathrm d} #2}}
\newcommand{\rmd}{ {\ \mathrm d} }
\newcommand{\uvec}[1]{ \hat{\mathbf #1} }
\newcommand{\pder}[2]{ \f{\partial #1}{\partial #2} }
\newcommand{\grad}{ {\bf \nabla } }
\newcommand{\curl}{ {\bf \nabla} \times}
\newcommand{\vol}{ {\mathcal V} }
\newcommand{\bndry}{ {\mathcal S} }
\newcommand{\dv}{~{\mathrm d}^3 x}
\newcommand{\da}{~{\mathrm d}^2 x}
\newcommand{\dl}{~{\mathrm d} l}
\newcommand{\dt}{~{\mathrm d}t}
\newcommand{\intv}{\int_{\vol}^{}}
\newcommand{\inta}{\int_{\bndry}^{}}
\newcommand{\avec}{ \vec A}
\newcommand{\ap}{ \vec A_p}

\newcommand{\bb}{\vec B}
\newcommand{\jj}{ \vec j}
\newcommand{\rr}{ \vec r}
\newcommand{\xx}{ \vec x}

% Definitions for the journal names
\newcommand{\adv}{    {\it Adv. Space Res.}} 
\newcommand{\annG}{   {\it Ann. Geophys.}} 
\newcommand{\aap}{    {\it Astron. Astrophys.}}
\newcommand{\aaps}{   {\it Astron. Astrophys. Suppl.}}
\newcommand{\aapr}{   {\it Astron. Astrophys. Rev.}}
\newcommand{\ag}{     {\it Ann. Geophys.}}
\newcommand{\aj}{     {\it Astron. J.}} 
\newcommand{\apj}{    {\it Astrophys. J.}}
\newcommand{\apjl}{   {\it Astrophys. J. Lett.}}
\newcommand{\apss}{   {\it Astrophys. Space Sci.}} 
\newcommand{\cjaa}{   {\it Chin. J. Astron. Astrophys.}} 
\newcommand{\gafd}{   {\it Geophys. Astrophys. Fluid Dyn.}}
\newcommand{\grl}{    {\it Geophys. Res. Lett.}}
\newcommand{\ijga}{   {\it Int. J. Geomagn. Aeron.}}
\newcommand{\jastp}{  {\it J. Atmos. Solar-Terr. Phys.}} 
\newcommand{\jgr}{    {\it J. Geophys. Res.}}
\newcommand{\mnras}{  {\it Mon. Not. Roy. Astron. Soc.}}
\newcommand{\nat}{    {\it Nature}}
\newcommand{\pasp}{   {\it Pub. Astron. Soc. Pac.}}
\newcommand{\pasj}{   {\it Pub. Astron. Soc. Japan}}
\newcommand{\pre}{    {\it Phys. Rev. E}}
\newcommand{\solphys}{{\it Solar Phys.}}
\newcommand{\sovast}{ {\it Soviet  Astron.}} 
\newcommand{\ssr}{    {\it Space Sci. Rev.}} 
\chardef\us=`\_

%%%%%%%%%%%%%%%%%%%%%%%%%%%%%%%%%%%%%%%%%%%%%%%%%%%%%%%%%%%%%%%%%%
\begin{document}

\begin{article}
\begin{opening}

\title{On the temperature of flaring active regions}

\author[addressref=aff1,corref,email={a.j.leonard@sheffield.ac.uk}]{\inits{A.J.}\fnm{Andrew}~\lnm{Leonard}}\sep
\author[addressref=aff2]{\inits{H.}\fnm{Huw}~\lnm{Morgan}}\sep
%%\author[addressref=aff1]{\inits{R.}\fnm{Robertus}~\lnm{Erd\'elyi}}
%\author{P.~\surname{Author-a}$^{1}$\sep
%        E.~\surname{Author-b}$^{1}$\sep
%        M.~\surname{Author-c}$^{2}$      
%       }

%   \institute{$^{1}$ First affiliation
%                     email: \url{e.mail-a} email: \url{e.mail-b}\\ 
%              $^{2}$ Second affiliation
%                     email: \url{e.mail-c} \\
%             }
\address[id=aff1]{Solar Physics \& Space Plasma Research Centre (SP$^{2}$RC), School of Mathematics and Statistics, The University of Sheffield, Hicks Building, Hounsfield Road, Sheffield, S3 7RH U.K.}
\address[id=aff2]{Institute of Mathematics, Physics and Computer Science, Prifysgol Aberystwyth, Ceredigion, Cymru SY23 3BZ, U.K.}

\runningauthor{Leonard et al.}
\runningtitle{On the temperature of flaring active regions}

\begin{abstract}
This paper presents a study which investigates the temperature distributions of several flaring active regions.
The aim is to search for a common signature of flare activity in these temperature distributions before flares occur, which, if found, could form the basis of a flare prediction algorithm.
This investigation looks at the temperature distributions of the active regions associated with two sets of flares, one set consisting of flares of various classes, and the other set being composed entirely of X-class flares.
For the first set, the mean temperature of each active region was compared to the class of the flare for a number of different times before the flare start time, which yielded no clear correlation.
The second set was investigated over the hour preceding each flare and a measure of the temperature variability was calculated.
This value was compared to the equivalent for a non-flaring active region observed over several hour-long windows.
In this case, it was found that the flaring active regions were typically much more variable than the non-flaring region used for comparison.
This was most clearly seen in the variability of the mean temperature, which exceeded the average variability of the non-flaring region mean temperature in all but one case.
This notable difference between the behaviour of the flaring regions and non-flaring region is worthy of further investigation, and demonstrates that this method could be used as part of a flare-prediction algorithm.
\end{abstract}
\keywords{}
\end{opening}
%-------------------------------------------------

\textbf{NOTE: All of the figures are going to be re-done and a few numbers might change slightly as part of that process. I'm not expecting this to change the conclusions at all.}

\section{Introduction}
\label{S-Introduction}

Solar flares are a complicated and as yet not fully understood phenomenon.
In particular, growing emphasis has been placed in recent years on studying how to predict when a flare will occur, since flares and associated coronal phenomena can have significant detrimental effects on a variety of modern electronic infrastructure at or near Earth.
Many studies have been devoted to the topic of flare prediction (e.g., \citet{Korsos2014,Ahmed2011,Bloomfield2012}, etc.), but few have looked at the temperature of the corresponding active regions before the flare occurs.

\subsection{Active region dynamics}
Solar flares are large releases of magnetic free non-potential energy due to magnetic reconnection. This release of energy results in rapid localised heating of plasma to several million Kelvins.
Due to their very high temperatures, flares emit very strongly in X-rays and EUV.
Although they do emit throughout the spectrum \citep{Fletcher2011}, only extremely large events can be seen in visible light even over the photospheric emission, such as the Carrington event of 1859 \citep{Carrington1859}.
Flares occur frequently in active regions which are developing rapidly, where flux is still emerging and the complexity of the magnetic topology increases very quickly, since the coronal field does not have time to reconfigure in a steady manner to rapid changes in the photosphere.
Free magnetic energy is therefore increased in the coronal field and the field becomes highly non-potential.
Reconnection leading to flares is triggered by wave disturbances or instabilities, and various models have been proposed which describe the magnetic topology required for flares to occur.
Large flares are the easiest to observe and the type we would most wish to predict, but similar, much smaller reconnection events have been observed called microflares and nanoflares.
These smaller events also release magnetic energy, which is thought by some to contribute to coronal heating .

\subsection{Temperature structure of active regions}
The temperature structure of active regions has been a topic of much debate for many years, and remains so primarily due to the intrinsic difficulties of measuring the temperature of plasma under such conditions.
However, it has been clear for a long time that active regions are the hottest large-scale features in the corona, with a sharp transition to much higher temperatures than their surroundings.
This was apparent from the first X-Ray observations of the Sun which showed strong emission from active regions.
DEM analyses of whole active regions have usually found temperatures around 1-2MK \citep{Brosius1996,Kashyap1998,Schmelz2009,Reale2009,Warren2009a,Goryaev2010}, though many have also reported finding a hotter plasma component at $\sim 5$-$10$ MK \citep{Brosius1996,Kashyap1998,Schmelz2009,Reale2009}.
These temperatures are typically interpreted as a diffuse background similar to the quiet sun, and hotter plasma within active region loops.
However, it has been suggested that some of the hotter temperatures found are due to reduced instrument sensitivity at lower temperatures (e.g.: \citet{Boerner2013}) or insufficient thermal coverage of emission lines at low temperatures.

As greater resolution was provided by new instruments, it became clear that active regions are composed of many individual large coronal loops.
Due to the low plasma-$\beta$ parameter of the corona, conduction across field lines is strongly restricted, and each coronal loop can therefore be viewed as an isolated system.
For this reason active region loops have often been treated as isothermal along their length, on the assumption that each loop has settled into thermal equilibrium.
Indeed, several studies have found active region loops to be isothermal or nearly isothermal (see, e.g.: \citet{Lenz1999,DelZanna2011,DelZanna2013}).
Many which do not report isothermal loops have found them to be cooler towards the footpoints (sometimes as cool as 0.5MK, e.g.: \citet{DelZanna2011}) and hottest at the apex of the loop \citep{Cheng1980,Aschwanden2000,DelZanna2003}.
Active region loops are often broadly categorised by their temperature into ``cool'' loops ($T\lesssim 1$ MK), ``warm'' loops ($T\approx 1$ MK) and ``hot'' loops ($T\gtrsim 1$ MK).
Any or all of these can be found within a single active region.

The current understanding is that many visible loops are not yet fully resolved and consist of many isothermal or nearly isothermal strands below the resolution limit of the current instrumentation.
These loops appear ``fuzzy'' in many images.
Indeed, the work of \citet{DelZanna2013} showed that such loops are still not resolved even at the higher resolution of the Hi-C instrument.
If these loops are multi-stranded, the strands must therefore be sufficiently fine that they are below even the resolution of Hi-C \citep{Cirtain2013}.
This being the case, temperature values currently accepted for these loops are likely to be incorrect and will need to be re-calculated when better instrumentation is available to resolve the currently sub-resolution loops in enough wavelengths.

\subsection{Temperatures of flares}
Solar flares are caused by magnetic reconnection, which can release large amounts of magnetic energy in a short space of time.
This causes plasma particles to be accelerated and very rapidly heat the plasma in the transition region and chromosphere, which then expands upwards into coronal loops \citep{Simoes2015} in a process called chromospheric evaporation.
Loops heated in this way have peak temperatures between 8 and 40 MK \citep{Ryan2013,Ryan2014}, and emit strongly in EUV and X-rays.
After this injection of heated material, these loops cool by conduction to the lower atmosphere until they reach temperatures of around 12MK \citep{Aschwanden2005}.
This is then followed by a long radiative cooling phase.

The peak temperatures of flares themselves vary between $\approx 5$ MK and $\approx 18$ MK with an average of $12$ MK for large (M and X class) flares.
Nanoflares are expected to produce heating to around $10$ MK \citep{Reale2011a}, but are difficult to observe because loops heated by nanoflares tend to have lower density and therefore much fainter emission than those heated by large flares.
Flares are inherently multi-thermal, but since their temperatures are typically calculated using instruments such as Hinode's X-Ray Telescope (XRT) and the Reuven Ramaty High Energy Solar Spectroscopic Imager (RHESSI), which are most sensitive to particularly high temperatures, values produced in this way tend to be biased towards the hotter end of a flare's temperature distribution.
Line- or filter-ratio methods have also often been used for temperature analysis, which have been shown to be strongly influenced by the temperature response of the instrument.
This again produces bias towards high temperature results when using instruments intended primarily to investigate hot plasma.
More recent studies which compare such results against temperatures calculated using AIA data (e.g.: \citet{Ryan2013,Ryan2014}) give a clearer idea of the full range of flare temperatures.

Though there have been many studies of temperature during flares, none have looked in detail at the temperatures, and changes in temperature, of active regions before the flare occurs.
This work aims to address this issue by searching for flare precursors in the temperature distributions of flaring active regions.

\section{Method}

The temperature map method described in \cite{Leonard} was used to investigate the changes in the temperatures over time of two sets of flaring active regions and one non-flaring active region.
Set 1 was composed of active regions associated with flares which occurred during February and March 2011, which had a variety of peak fluxes.
The flares detected included many small flares (B and C class) as well as several larger M class flares and two X class flares.
This set was used to investigate the possibility of a link between active region temperature and flare peak flux.
Set 2 spanned a much longer period of time, between 2010-07-14 and 2011-09-19, and included only active regions associated with X class flares.
The temperatures of this set were compared against those of the non-flaring active region in order as a measure of whether large flares produce any visible thermal signature before the flare onset.
The non-flaring region is used to determine the ``normal'' background variation in active region temperature in the absence of flares.

\subsection{Set 1: Flares of different classes}
For each flare event investigated in Set 1, a temperature map was calculated for a 150 x 150 arcsec area centred on the corresponding active region at four different times: 30 minutes before the flare, 10 minutes before the flare, 1 minute before the flare, and at the flare onset time.
For each time interval the peak fluxes of the flares were plotted against the mean temperatures of the associated active regions.
The start times of the flares and the locations of the active regions on the solar disk were obtained by querying the Heliophysics Events Knowledgebase (HEK), a database of solar features and events.
These queries were performed using SunPy's wrappers around the HEK, allowing the process to be automated.
An example active region temperature map is shown in Fig \ref{fig:ar-demo}.
The specific flares investigated are listed in Appendix A along with the active region associated with each. % Put the appendix in and cite this properly

\begin{figure}
  \centerline{\includegraphics[width=0.6\textwidth]{../figs/2011-03-09T2240_fulldisk}}
  \centerline{\includegraphics[width=0.6\textwidth]{../figs/2011-03-09T2240}}
\caption{Example of an active region temperature map used for this investigation, and its location on the solar disk.
Left: full-disk AIA 17.1nm image.
The blue box indicates the boundaries of the region shown in the temperature map on the right.
Right: close-up temperature map of the active region studied.
The box indicates the portion of the map within which the mean temperature was calculated.
The location of each active region investigated was determined automatically by querying the HEK through SunPy.\label{fig:ar-demo}}
\end{figure}

\subsection{Set 2: X class flares}
150 x 150 arcsec temperature maps were calculated of the active regions associated with the set 2 flares, in the same way as for the first set.
In this case the active regions were mapped at 2-minute intervals for 30 minutes preceding the start of the flare.
The maximum, 95th percentile, 90th percentile, mean, standard deviation, 10th percentile, 5th percentile and minimum temperatures were then calculated for each temperature map for each flare in order to compare if and how the bulk temperature of the active regions changed before the flare.
The 95th, 90th, 10th and 5th percentiles were calculated because the overall maximum and minimum may be affected by small numbers of unusually high or low temperature pixels, which may not accurately represent how the bulk temperature of the active region changes with time.
Each of these parameters was plotted against time for each active region.
Additionally, for each active region the change in the maximum temperature over time, $\frac{\Delta T_{max}}{\Delta t}$, was calculated for each time step, and the average of this value was taken over the whole 30-minute observation of the active region.
The values for $\left\langle \frac{\Delta T_{max}}{\Delta t}\right\rangle$ and the times of the associated flares are shown in Table 2 for each active region in set 2. % Check this should be max

The parameters used to investigate the set 2 active regions were also calculated and plotted for a non-flaring active region, AR11268, in order to compare how the change in temperature distributions differed between flaring and non-flaring regions.
AR11268 was observed to almost complete a full transit of the solar disk between 2011-08-04 and 2011-08-14.
During this time, temperature maps were calculated at 2-minute resolution for windows of 60 minutes once a day, between 00:00 and 00:58. $\left\langle \frac{\Delta T_{max}}{\Delta t}\right\rangle$ values were calculated for each 60-minute period in the same way as for the flaring active regions.
Table 3 shows the times of the observations and the values of $\left\langle \frac{\Delta T_{max}}{\Delta t}\right\rangle$.

\section{Results}

\subsection{Flare flux vs temperature}
Plots of the peak flux against active region mean temperature are shown in Fig \ref{fig:flux-v-temp} for each flare in set 1.
For the vast majority of these flares there is no clear correlation, most active regions being distributed seemingly at random in the temperature range between $log(T)\approx6.0$ and $log(T)\approx6.15$, with a few outliers below $log(T)=6.0$.
A small number of the largest flares ($\log(\text{flux}) > -4.5$) do appear to show a loose trend of increasing flux with temperature, but no conclusions can be drawn from this since such a trend is not seen for any of the other flares in this set.

Heating and cooling in each active region with respect to the previous time (i.e., from 30 minutes to 10 minutes before the flares, from 10 minutes to 1 minute, and from 1 minute to the flare start time) are indicated by red and blue arrows respectively on the top right and bottom plots in Fig \ref{fig:flux-v-temp}.
These arrows indicate that the mean temperatures of the active regions are quite stable over the 30 minutes studied.
Even between 30 minutes and 10 minutes before the flare - the largest time difference - most changes in temperature are quite small.
A few active regions cool significantly in this time, and one of these appears to reheat to roughly its original temperature by 1 minute before the flare.
Many active regions appear to show no heating or cooling at all between any of the times plotted.
What temperature changes there are are split roughly evenly between temperature increases and decreases, showing no obvious signs of systematic heating or cooling at any point before the flare.

\begin{figure}
\begin{centering}
\includegraphics[width=1\textwidth]{../figs/allflares} 
\par\end{centering}

\caption{Peak flux of each flare in set 1 plotted against the mean temperature of the corresponding active region at several different times.
Top left: 30 minutes before the flare.
Top right: 10 minutes before the flare.
Bottom left: 1 minute before the flare.
Bottom right: at the start time of the flare.
The red and blue arrows on each plot indicate whether an active region's mean temperature has increased or decreased respectively, with respect to the previous plot.
The dashed lines indicate flux thresholds for (from bottom to top) B, C, M and X class flares.\label{fig:flux-v-temp}}
\end{figure}

\subsection{Temperature change over time}
\subsubsection{Non-flaring active region}
The plots in Fig \ref{fig:ar11268-allparams} show the variation in the minimum, 5th percentile, 10th percentile, mean 90th percentile, 95th percentile and maximum temperatures for active region AR11268 during several 60-minute windows.
During each window, the temperature remains very stable, showing almost no variation for any of the parameters.
The minimum temperature was usually between $log(T)=5.95$ and $6.0$, and was slightly below $log(T)=5.95$ during one of the 60-minute windows.
Unlike the minima of several of the flaring active regions (see next subsection), the AR11268 minima show no signs of dropping to $log(T)=5.6$ at any point during any of the times the region was observed.
The 5th, 10th, 90th and 95th percentiles were all very stable as well, in some cases remaining constant for the full hour.
5th and 10th percentiles were usually at or slightly below $log(T)=6.0$, while 90th and 95th percentiles were usually at or slightly below $log(T)=6.1$.
The mean was $log(T)\approx6.05$ for all windows, usually slightly cooler, and shows almost no discernible change in any of the plots.
The maximum, like the minimum, was much more stable than those of the flaring regions, and never rose above $log(T)\approx6.2$.
For most of the duration of the active region it was below $log(T)\approx6.15$.

Overall, most of the temperature parameters for the non-flaring active region AR11268 are very similar to those seen for the flaring active regions shown below.
The flaring active regions tend to have slightly higher temperatures, but all of them are similarly uniform with time and in all cases the temperatures are quite closely distributed.
The main difference seen between the flaring regions and AR11268 is the large jumps seen in the minimum and maximum temperatures of the former, which are not seen at all for AR11268.
This is particularly obvious for the maximum, which displays these jumps more frequently than the minimum in flaring regions.
By contrast, the maximum of AR11268 shows almost no change during several of the windows studied, and always remains within the range $log(T)\approx6.1-6.2$. 

The mean time variability of the temperature, $\left\langle \frac{\Delta T}{\Delta t}\right\rangle$, was calculated for each 60-minute window for the 5th and 95th percentiles and the mean.
These values are shown in Table \ref{tab:dTdt-11268}, and are plotted as a bar graph in Fig \ref{fig:Bar-graph-quiet}.
Also shown in Fig \ref{fig:Bar-graph-quiet} are the mean over all windows of the mean time variability for each parameter (which we take to be a background level of variability), and this value plus one, two and three standard deviations.
From these plots it can be seen that none of the windows' variability is greater than three standard deviations above the background, and only a few exceed a single standard deviation.

\begin{figure}
\begin{centering}
\includegraphics[height=0.8\textheight]{../figs/ar11268} 
\par\end{centering}

\caption{Plots of all parameters for active region AR11268.
Colour-coding of the lines is the same as for Figs \ref{fig:temps-v-time-xflares} and \ref{fig:temps-v-time-xflares-1}. % Sort out ordering of these
Each plot shows how these temperatures change during a 60-minute window, each separated by a day.
All of the temperature parameters vary only very slightly during each window, showing a very stable distribution overall.
The maximum and minimum temperatures here are noticeably more stable than for many of the flaring active regions, and remain entirely within the temperature range plotted for all times investigated.\label{fig:ar11268-allparams}}
\end{figure}

\begin{table}
\caption{Mean variability of temperature with time for the 5th and 95th percentiles (robust minimum and maximum) and the mean temperature for each of the observation windows of active region AR11268.
The robust minimum and maximum are less predictable than the mean, showing no variation at all in some windows and relatively large amounts in others.
The mean, on the other hand, shows a consistent low level of variation.
Note that the window for 2011-08-12 is absent, since temperature maps could not be calculated for the full window.\label{tab:dTdt-11268}}
\centering{}\begin{tabular}{c|c|c|c|c}
Date  & $\langle\frac{\Delta T_{5th}}{\Delta t}\rangle$  & $\langle\frac{\Delta T_{mean}}{\Delta t}\rangle$  & $\langle\frac{\Delta T_{95th}}{\Delta t}\rangle$  & \selectlanguage{british}%
\selectlanguage{british}%
\tabularnewline
\hline 
2011-08-04  & $1.724095\times10^{-04}$  & $8.517236\times10^{-05}$  & $0.000000\times10^{+00}$  & \selectlanguage{british}%
\selectlanguage{british}%
\tabularnewline
2011-08-05  & $3.448355\times10^{-04}$  & $9.477340\times10^{-05}$  & $3.448355\times10^{-04}$  & \selectlanguage{british}%
\selectlanguage{british}%
\tabularnewline
2011-08-06  & $0.000000\times10^{+00}$  & $9.199770\times10^{-05}$  & $1.724095\times10^{-04}$  & \selectlanguage{british}%
\selectlanguage{british}%
\tabularnewline
2011-08-07  & $9.999752\times10^{-04}$  & $1.015857\times10^{-04}$  & $0.000000\times10^{+00}$  & \selectlanguage{british}%
\selectlanguage{british}%
\tabularnewline
2011-08-08  & $0.000000\times10^{+00}$  & $8.772782\times10^{-05}$  & $1.724095\times10^{-04}$  & \selectlanguage{british}%
\selectlanguage{british}%
\tabularnewline
2011-08-09  & $0.000000\times10^{+00}$  & $1.334918\times10^{-04}$  & $0.000000\times10^{+00}$  & \selectlanguage{british}%
\selectlanguage{british}%
\tabularnewline
2011-08-10  & $1.724095\times10^{-04}$  & $1.299373\times10^{-04}$  & $3.448190\times10^{-04}$  & \selectlanguage{british}%
\selectlanguage{british}%
\tabularnewline
2011-08-11  & $0.000000\times10^{+00}$  & $9.097693\times10^{-05}$  & $0.000000\times10^{+00}$  & \selectlanguage{british}%
\selectlanguage{british}%
\tabularnewline
2011-08-13  & $0.000000\times10^{+00}$  & $1.102278\times10^{-04}$  & $5.172532\times10^{-04}$  & \selectlanguage{british}%
\selectlanguage{british}%
\tabularnewline
\hline 
Avg over all windows  & $1.877366\times10^{-04}$  & $1.028768\times10^{-04}$  & $1.724141\times10^{-04}$ & \selectlanguage{british}%
\tabularnewline
\end{tabular}\selectlanguage{british}%


\end{table}

\begin{figure}
\begin{centering}
\includegraphics[width=0.32\textwidth]{../figs/dTdt-11268-5th}\includegraphics[width=0.32\textwidth]{figs/dTdt-11268-mean}\includegraphics[width=0.32\textwidth]{figs/dTdt-11268-95th}
\par\end{centering}

\caption{Bar graph of $\langle\frac{\Delta T_{5th}}{\Delta t}\rangle$ (left), $\langle\frac{\Delta T_{mean}}{\Delta t}\rangle$ (centre) and $\langle\frac{\Delta T_{95th}}{\Delta t}\rangle$ (right), as shown in Table \ref{tab:dTdt-11268}, for all of the time windows during which the non-flaring active region was mapped.
On each plot the yellow dashed line shows the average value over all windows, and the light orange, dark orange and red lines indicate the mean plus one, two and three times the standard deviation over all windows, respectively.
Percentage values in the legend indicate the proportion of windows in which the variability exceeds that amount.\label{fig:Bar-graph-quiet}}
\end{figure}

\subsubsection{Flaring active regions}
Figs \ref{fig:temps-v-time-xflares} and \ref{fig:temps-v-time-xflares-1} show how the temperature properties of the active region changed during the 30 minutes preceding each flare in set 2.
Each flare is plotted separately but on the same temperature scale so that any changes common to many flares can be easily noticed.
In each plot, the maximum is plotted in black, the 95th percentile in orange, the 90th percentile in purple, the mean in light blue, with error bars indicating the standard deviation, the 10th percentile in red, the 5th percentile in green and the minimum in dark blue.
The full temperature range of the parameters is plotted in the graphs shown in Fig \ref{fig:temps-v-time-xflares}, but details of the mean and percentiles are difficult to discern due to the wide range of the minimum and maximum.
Fig \ref{fig:temps-v-time-xflares-1} therefore shows the same plots as Fig \ref{fig:temps-v-time-xflares}, but with a smaller temperature range so that small changes in the mean and percentiles can more easily be seen.

The maximum temperature of several of the active regions is steady at slightly below $log(T)=6.2$ for the entire 30 minutes plotted, and varies only by very small amounts during this time.
Some other active regions show a similarly flat profile with time, but at $log(T)\gtrsim6.4$.
The remainder appear to switch sharply between these two regimes, with varying frequencies.
The maxima of these regions are a little more variable than the others, particularly within the hotter temperature range, between $log(T)\approx6.4$ and $log(T)=6.6$.
These hotter maxima are significantly hotter than the 95th and 90th percentiles, which show no similar jumps when the maximum changes.
This suggests that the maximum temperature must be found only in one or a very small number of pixels, so that the mean and percentiles are not affected.

The 95th and 90th percentiles are very similar for all active regions, and in some cases appear to be identical for much of the duration.
This demonstrates a very narrow distribution of temperatures.
Both are typically between $log(T)=6.1$ and $6.15$, only slightly cooler than the cool maxima, and again show very little variation over the 30 minutes leading up to the flare.

Mean temperatures are uniform, with each active region showing almost no change in the mean temperature at all throughout the 30 minutes before the flare.
A few active regions show a very small increase or decrease in the mean, typically within the 15 minutes or so before the flare.
Almost every active region had a mean between $log(T)=6.05$ and $6.1$.

The 10th and 5th percentiles (robust minima) are also very uniform, remaining constant for several active regions and changing only very slightly for the rest.
They are typically both between $log(T)=6.0$ and $6.05$, although the 10th percentile is slightly above $log(T)=6.05$ for a few particularly hot active regions.

The minimum temperatures show a similar behaviour to the maxima, though to a lesser extent.
For the most part the minimum of each active region is between $log(T)=5.95$ and $6.0$, and changes very little.
However, for some active regions it does sharply transition to $log(T)=5.6$, the minimum possible value with this method.
This very low minimum is usually only found for a single temperature map at a time, but there are also some extended periods of time where it is found.
As with the maximum temperatures, this extremely low value appears not to have any noticeable effect on the mean or percentiles in any active region, suggesting it must be true for one or only a few pixels.

The mean variability of the robust minima and maxima and the mean are shown in Table \ref{tab:dTdt} and plotted in Fig \ref{fig:Bar-graph-flaring}.
The dashed lines in these plots again show the means of the variability of AR11268 and the mean plus one, two and three standard deviation in all windows.
The robust minima and maxima show only slightly more variability than the non-flaring region, with less than a third of active regions in each case reaching a standard deviation above the background level.
The mean, however, is above the background in all but one active region, with almost half of the active regions exceeding the background by $1\sigma$ and one region exceeding it by $3\sigma$.

\begin{figure}
\begin{centering}
\includegraphics[width=0.7\textwidth]{../figs/allars_0} 
\par\end{centering}

\caption{Plots of the minimum, (dark blue lines) 5th percentile (green lines), 10th percentile (red), mean (light blue), 90th percentile (purple), 95th percentile (orange) and maximum (black) temperatures for the active region associated with the first six flares in set 2 for the 30 minutes leading up to the flare start time, which is indicated for each flare.
Error bars on the mean indicate the standard deviation.
In all cases the mean, standard deviation and percentiles vary very little throughout the sampled time and are closely grouped between $log(T)=5.95$ and $6.2$.
The minima are typically around $log(T)=6.0$ and also varies only a small amount for the most part, but does in some cases sharply switch to $log(T)=5.6$, the lowest temperature possible with this method.
Maximum temperatures similarly switch between relatively low values and much higher ranges, where they are slightly more changeable than the minimum values.\label{fig:temps-v-time-xflares}}
\end{figure}

\begin{figure}
\begin{centering}
\includegraphics[width=0.8\textwidth]{../figs/allars_1} 
\par\end{centering}

\caption{Same plots as Fig \ref{fig:temps-v-time-xflares} for the next six flares.
\label{fig:temps-v-time-xflares-2}}
\end{figure}

\begin{figure}
\begin{centering}
\includegraphics[width=0.8\textwidth]{../figs/allars_2} 
\par\end{centering}

\caption{Same plots as Fig \ref{fig:temps-v-time-xflares-2} for the next six flares.
\label{fig:temps-v-time-xflares-3}}
\end{figure}

\begin{figure}
\begin{centering}
\includegraphics[width=0.8\textwidth]{../figs/allars_3} 
\par\end{centering}

\caption{Same plots as Fig \ref{fig:temps-v-time-xflares-3} for the next four flares.\label{fig:temps-v-time-xflares-4}}
\end{figure}

\begin{figure}
\begin{centering}
\includegraphics[width=0.8\textwidth]{../figs/allars_close_0} 
\par\end{centering}

\caption{Plots of the minimum, 5th percentile, 10th percentile, mean 90th percentile, 95th percentile and maximum temperatures for the active region associated with each of the first six flares of set 2.
Plots are the same as Fig \ref{fig:ar11268-allparams}, but restricted to between $log(T)=5.9$ and $6.2$ so small variations in the mean and percentiles can be seen more clearly.\label{fig:temps-v-time-xflares-1}}
\end{figure}

\begin{figure}
\begin{centering}
\includegraphics[width=0.8\textwidth]{../figs/allars_close_1} 
\par\end{centering}

\caption{Same plots as Fig \ref{fig:temps-v-time-xflares-1} for the next six flares.
\label{fig:temps-v-time-xflares-1-1}}
\end{figure}


\begin{figure}
\begin{centering}
\includegraphics[width=0.8\textwidth]{../figs/allars_close_2} 
\par\end{centering}

\caption{Same plots as Fig \ref{fig:temps-v-time-xflares-1-1} for the next six flares.\label{fig:temps-v-time-xflares-1-2}}
\end{figure}

\begin{figure}
\begin{centering}
\includegraphics[width=0.8\textwidth]{../figs/allars_close_3} 
\par\end{centering}

\caption{Same plots as Fig \ref{fig:temps-v-time-xflares-1-2} for the next four flares.\label{fig:temps-v-time-xflares-1-3}}
\end{figure}

\begin{table}
\caption{Mean variability of temperature with time for the 5th and 95th percentiles (robust minimum and maximum) and the mean temperature for each of the observed active regions in set 2.
As with the non-flaring region, the variabilities of the robust minimum and maximum are very high for some regions and zero for others.
Again, similar to AR11268, the mean is more uniform from one region to the next, making it of greater use for predictive purposes.\label{tab:dTdt}}
%\centering{}\selectlanguage{english}%
\begin{tabular}{c|c|c|c|c}
Date and time  & $\langle\frac{\Delta T_{5th}}{\Delta t}\rangle$  & $\langle\frac{\Delta T_{mean}}{\Delta t}\rangle$  & $\langle\frac{\Delta T_{95th}}{\Delta t}\rangle$  & \selectlanguage{british}%
\selectlanguage{british}%
\tabularnewline
\hline 
2010-07-14 12:04:50  & $0.000000\times10^{+00}$  & $1.543411\times10^{-04}$  & $0.000000\times10^{+00}$  & \selectlanguage{british}%
\selectlanguage{british}%
\tabularnewline
2010-07-17 18:04:42  & $3.571510\times10^{-04}$  & $3.895366\times10^{-04}$  & $1.428587\times10^{-03}$  & \selectlanguage{british}%
\selectlanguage{british}%
\tabularnewline
2010-11-13 11:36:34  & $0.000000\times10^{+00}$  & $1.487640\times10^{-04}$  & $0.000000\times10^{+00}$  & \selectlanguage{british}%
\selectlanguage{british}%
\tabularnewline
2011-01-24 05:58:14  & $5.357265\times10^{-04}$  & $2.141391\times10^{-04}$  & $6.249845\times10^{-04}$  & \selectlanguage{british}%
\selectlanguage{british}%
\tabularnewline
2011-02-15 01:44:18  & $0.000000\times10^{+00}$  & $3.332069\times10^{-04}$  & $3.571340\times10^{-04}$  & \selectlanguage{british}%
\selectlanguage{british}%
\tabularnewline
2011-02-16 23:56:54  & $3.571340\times10^{-04}$  & $2.425750\times10^{-04}$  & $3.571340\times10^{-04}$  & \selectlanguage{british}%
\selectlanguage{british}%
\tabularnewline
2011-03-01 07:02:38  & $0.000000\times10^{+00}$  & $3.502850\times10^{-04}$  & $3.571340\times10^{-04}$  & \selectlanguage{british}%
\selectlanguage{british}%
\tabularnewline
2011-03-04 01:12:14  & $7.142680\times10^{-04}$  & $5.923882\times10^{-04}$  & $1.071419\times10^{-03}$  & \selectlanguage{british}%
\selectlanguage{british}%
\tabularnewline
2011-03-09 23:10:02  & $0.000000\times10^{+00}$  & $1.355973\times10^{-04}$  & $0.000000\times10^{+00}$  & \selectlanguage{british}%
\selectlanguage{british}%
\tabularnewline
2011-03-30 19:56:10  & $0.000000\times10^{+00}$  & $3.017686\times10^{-04}$  & $1.071402\times10^{-03}$  & \selectlanguage{british}%
\selectlanguage{british}%
\tabularnewline
2011-04-15 03:40:30  & $0.000000\times10^{+00}$  & $1.098131\times10^{-04}$  & $0.000000\times10^{+00}$  & \selectlanguage{british}%
\selectlanguage{british}%
\tabularnewline
2011-04-16 03:04:22  & $0.000000\times10^{+00}$  & $1.903664\times10^{-04}$  & $3.571510\times10^{-04}$  & \selectlanguage{british}%
\selectlanguage{british}%
\tabularnewline
2011-04-22 01:04:58  & $0.000000\times10^{+00}$  & $4.161398\times10^{-04}$  & $7.142680\times10^{-04}$  & \selectlanguage{british}%
\selectlanguage{british}%
\tabularnewline
2011-04-22 18:56:38  & $3.571510\times10^{-04}$  & $3.684770\times10^{-04}$  & $1.071402\times10^{-03}$  & \selectlanguage{british}%
\selectlanguage{british}%
\tabularnewline
2011-06-02 08:38:46  & $0.000000\times10^{+00}$  & $1.214246\times10^{-04}$  & $0.000000\times10^{+00}$  & \selectlanguage{british}%
\selectlanguage{british}%
\tabularnewline
2011-06-03 16:06:26  & $0.000000\times10^{+00}$  & $1.161367\times10^{-04}$  & $0.000000\times10^{+00}$  & \selectlanguage{british}%
\selectlanguage{british}%
\tabularnewline
2011-06-03 22:30:18  & $3.571340\times10^{-04}$  & $2.205662\times10^{-04}$  & $0.000000\times10^{+00}$  & \selectlanguage{british}%
\selectlanguage{british}%
\tabularnewline
2011-07-28 00:12:42  & $0.000000\times10^{+00}$  & $2.846165\times10^{-04}$  & $1.071402\times10^{-03}$  & \selectlanguage{british}%
\selectlanguage{british}%
\tabularnewline
2011-07-30 02:04:46  & $3.571340\times10^{-04}$  & $9.399487\times10^{-05}$  & $0.000000\times10^{+00}$  & \selectlanguage{british}%
\selectlanguage{british}%
\tabularnewline
2011-09-06 22:08:34  & $7.143021\times10^{-04}$  & $2.344959\times10^{-04}$  & $3.571340\times10^{-04}$  & \selectlanguage{british}%
\selectlanguage{british}%
\tabularnewline
2011-09-13 22:46:06  & $0.000000\times10^{+00}$  & $1.254477\times10^{-04}$  & $0.000000\times10^{+00}$  & \selectlanguage{british}%
\selectlanguage{british}%
\tabularnewline
2011-09-18 09:44:26  & $0.000000\times10^{+00}$  & $1.117595\times10^{-04}$  & $0.000000\times10^{+00}$  & \selectlanguage{british}%
\selectlanguage{british}%
\tabularnewline
2011-09-19 07:54:14  & $0.000000\times10^{+00}$  & $0.000000\times10^{+00}$  & $0.000000\times10^{+00}$  & \selectlanguage{british}%
\selectlanguage{british}%
\tabularnewline
2011-09-19 08:08:22  & $0.000000\times10^{+00}$  & $0.000000\times10^{+00}$  & $0.000000\times10^{+00}$  & \selectlanguage{british}%
\selectlanguage{british}%
\tabularnewline
\hline 
Avg over all regions  & $1.442308\times10^{-04}$  & $2.021477\times10^{-04}$  & $3.399674\times10^{-04}$ & \selectlanguage{british}%
\selectlanguage{british}%
\tabularnewline
\end{tabular}\selectlanguage{british}%

 
\end{table}

\begin{figure}
\begin{centering}
\includegraphics[width=0.32\textwidth]{../figs/dTdt-5th}\includegraphics[width=0.32\textwidth]{figs/dTdt-mean}\includegraphics[width=0.32\textwidth]{figs/dTdt-95th}
\par\end{centering}

\caption{Bar graph of $\langle\frac{\Delta T_{5th}}{\Delta t}\rangle$ (left), $\langle\frac{\Delta T_{mean}}{\Delta t}\rangle$ (centre) and $\langle\frac{\Delta T_{95th}}{\Delta t}\rangle$ (right), as shown in Table \ref{tab:dTdt}, for all active regions in set 2.
On each plot the yellow dashed line shows the average of the respective value for the non-flaring region over all windows investigated, and the light orange, dark orange and red lines indicate the mean plus one, two and three times the standard deviation over all windows, respectively.\label{fig:Bar-graph-flaring}}
\end{figure}

\section{Discussion}

It should be noted that this method does not take into account several factors pertinent to determination of active region temperatures.
First, the effect of LOS integration is not corrected for in any way, such as with background subtraction.
This will be less of a problem while active regions are on the solar disk (when the measured LOS emission will be dominated by the active regions loops) than when they are over the limb where there is significant background contamination, but it should still be considered when reviewing these results.
The method is also applied under that assumptions that the plasma being studied is in local thermal equilibrium and that the plasma DEM has a narrow Gaussian distribution.
Both of these assumptions are potential causes of error in active regions.
For more detail on the method and its limitations, see \cite{Leonard}.
%However, if these assumptions hold then the results should be reliable as demonstrated by the validation in \ref{chap:Validation}. % Ref to thesis? or just state this?

\subsection{Temperatures of flaring active regions}
From this study, it appears that there is no immediately clear link between solar flares and the evolution of active region temperatures prior to them flaring.
However, there are some promising lines of inquiry which should be investigated further.
Figs \ref{fig:temps-v-time-xflares} and \ref{fig:temps-v-time-xflares-1} show that there is no clear trend in any of the temperature parameters investigated before large flares occur, though several active regions show large jumps in the minimum and maximum temperatures to very low and high temperatures.
These large jumps are not present in the non-flaring active region investigated, AR11268, which suggests a possible way in which flaring and non-flaring regions which may be distinguished before flares occur.
This link is further demonstrated by the values for the average rate of change of the maximum temperature, $\left\langle \frac{\Delta T_{max}}{\Delta t}\right\rangle $, which shows extremely slow change for AR11128 but much faster change for many of the flaring active regions.
However, this may be of limited usefulness since these sharp transitions only occur in some of the flaring active regions but not in others.
AR11268 is also slightly cooler than most of the flaring active regions.
More non-flaring regions should be investigated to confirm these differences.

Similarly, no definitive correlation can be seen between peak flare flux and active region temperature, with most flares being quite evenly distributed within a certain range of temperatures.
However, the largest flares do appear to form a loose correlation.
The coolest active regions studied may also be be part of the same progression, though they are noticeably removed from the other points on the plot so it is possible they are merely outliers.
However, if they are not outliers, the line these points form with the largest flares may describe a maximum peak flux attainable for flares from an active region at a given temperature.
If such a relation exists it would make an extremely useful part of a flare prediction algorithm, but a much larger sample of flares is needed in order to properly determine whether or not this connection is real.

\section{Conclusion} % Merge into section 4?

Although the results of this study are not conclusive, they do suggest a number of promising directions for future research.
The mean active region temperature and peak flare flux mostly have no clear correlation, but may show some very loose trends over a few small temperature ranges.
This needs to be investigated further with a much larger sample of flares.
If these trends are shown to be real with a larger sample size, they may describe a maximum peak flare flux for an active region at a given temperature, which would prove to be a very useful tool in predicting large events.

Similarly, the variability over time of flaring active regions' temperature distributions shows the potential for predictive capability but needs a much larger sample size and a more rigorous definition of the background variability.
The variability in the mean temperature of the regions studied, while small, is significantly greater than that of the non-flaring region investigated, exceeding the latter's mean variability by a standard deviation in $45.5\%$ of cases.

This investigation demonstrates that the extremely fast, automated temperature analysis provided by the temperature map method can be a very powerful tool for probing dynamic processes in the corona.

%%%% BIBLIOGRAPHY %%%%%%%%%%%%%%%%%%%%%%%%%%%%%%%%%%%%%%%%%%%%%%%%%%%%%%%%%%%
     % format of references provided by the journal (.bst)
\bibliographystyle{spr-mp-sola}
     % name your Bibtex file containing your references (.bib)
\bibliography{library}

     % Checking: look if the file containing the ``\bibitem'' exits
     %           so check if the .bbl file exist (bibTeX compilation)
\IfFileExists{\jobname.bbl}{} {\typeout{}
\typeout{****************************************************}
\typeout{****************************************************}
\typeout{** Please run "bibtex \jobname" to obtain} \typeout{**
the bibliography and then re-run LaTeX} \typeout{** twice to fix
the references !}
\typeout{****************************************************}
\typeout{****************************************************}
\typeout{}}

\end{article} 

\end{document}
